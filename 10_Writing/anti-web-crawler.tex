\documentclass[sigconf,anonymous=true]{acmart}

\usepackage{booktabs} % For formal tables

% Copyright
\setcopyright{none}
%\setcopyright{acmcopyright}
%\setcopyright{acmlicensed}

%\setcopyright{rightsretained}

%\setcopyright{usgov}
%\setcopyright{usgovmixed}
%\setcopyright{cagov}
%\setcopyright{cagovmixed}

\settopmatter{printacmref=false}



% DOI
\acmDOI{}

% ISBN
\acmISBN{}

%Conference
\acmConference[ACSAC'18]{Annual Computer Security Applications Conference}{December 2018}{San Juan, Puerto Rico}
\acmYear{2018}
\copyrightyear{2018}

\acmArticle{}
\acmPrice{15.00}

% These commands are optional
%\acmBooktitle{Transactions of the ACM Woodstock conference}
\editor{}
\editor{}
\editor{}


\begin{document}
\title{Anti-Methods for Distributed Web-Crawler:\\ Long-tail Threshold Model}
%\titlenote{Produces the permission block, and copyright information}

\author{Inwoo Ro}
\authornote{Inwoo Ro insisted his name be first.}
\orcid{1234-5678-9012}
\affiliation{%
  \institution{Hanyang University}
  \streetaddress{P.O. Box 1212}
  \city{Seoul}
  \country{Korea}
  \postcode{43017-6221}
}
\affiliation{%
  \institution{NAVER WEBTOON Corp.}
  \streetaddress{P.O. Box 1212}
  \city{Bundang}
  \country{Gyeoungi-do}
  \postcode{43017-6221}
}
\email{inwoo13@hanyang.ac.kr}

\author{Joong Soo Han}
\authornote{The secretary disavows any knowledge of this author's actions.}	
\affiliation{%
  \institution{Hanyang University}
  \city{Seoul}
  \country{Korea}
  \postcode{43017-6221}
}
\email{soohan@hanyang.ac.kr}

\author{Eul Gyu Im}
\authornote{This author is the
  one who did all the really hard work.}
\affiliation{%
  \institution{Hanyang University}
  \city{Seoul}
  \country{Korea}
  \postcode{43017-6221}
  }
\email{imeg@hanyang.ac.kr}

% The default list of authors is too long for headers.
%\renewcommand{\shortauthors}{B. Trovato et al.}


\begin{abstract}
In this paper, we propose a countermeasure against distributed crawlers. We first introduce known crawler detection methods in general and how distributed crawlers bypass these detection methods. Then, we propose a new method that can detect distributed crawlers by focusing on the property that web traffics follow the power distribution. This means when we sort the items by the number of requests, most of the requests are concentrated on the most frequently requested items.[1] And there will be a long-tail area that legitimate users do not generally request but crawlers will. Because crawler algorithms are generally intend collect every target items. So crawlers request iteratively and recursively request items by parsing web service architecture. By following these two assumptions, we can assume that if some IPs are frequently requesting the items that are located in longtail area, those IPs could be classified as crawler nodes. We tested this theory by simulating with real world web traffic data that NASA released and found the effective and low false positive results.
\end{abstract}


\keywords{Web Crawler, Traffic Analysis, Power Law, Information Theory}

\maketitle


%
% Introduction
%
\section{Introduction}
Web-crawling is used in various fields for collecting data.[8] Some web-crawlers collect data even though the target site is prohibiting crawlers by robot.txt. And even though web services try to detect and prevent crawlers by anti-crawler methods, these malicious web-crawlers bypass detection by modifying their header values or distributing IPs to masquerade as if they are legitimated users.
It is a matter of availability and data property issue because even though any service permits viewing of each data, it is prohibited to duplicate an entire dataset. And malicious web-crawlers can cause significant traffic and intellectual property infringement by collecting the entire data from web services. This can have a serious impact on the availability of the target service.
In this paper, we introduce the conventional anti-crawling methods and its countermeasures, and show that the conventional anti-crawling methods cannot defend against the distributed crawler. Then we propose the new anti-crawling technique that we call 'node-reducing' which gradually adds the distributed crawler node IPs to the block-list.

%
% Back Ground
% with 2 subsections
%
\section{Related Works}
In this section, we will describe about conventional anti-crawling methods and their counter crawling measures.
\begin{enumerate}
\item HTTP Header Check
\newline A basic crawler will send requests without modifying its header information. And web servers can distinguishes a legitimate user from a crawler by checking the request header, especially User-Agent value has been set properly. This header checking method is a basic anti-crawling method.
But if a crawler attempts to masquerade itself as a legitimate user, it will replay the header information from the web browser or form the http header information similar to a browser. This makes it difficult for a web server to determine whether a client is a crawler or a legitimate user by simply checking the request header.
\newline

\item Access Pattern based Anti-Crawling
\newline
Access pattern based anti-crawling is a method of classifying legitimate users and crawlers based on the pattern requested by the client. If a client requests only a specific resource continuously without a call to a resource that should normally be requested, the corresponding could be regarded as a crawler. An attacker performing an aggressive crawling predefines the core data that the web service wants to collect, and implements a crawler that requests specific data without requesting unnecessary resources. In this case, the web server can recognize that the client is not a legitimate user. In the case of a web service using advanced approach to access pattern recognition, the service is viewed as a set of consecutive requests from the viewpoint of the user UX, and requests and responses belonging to the same set are chained by including a specific hash value in the cookie. Although this approach can recognize a crawler based on access pattern, some crawlers even masquerade their access pattern by analyzing network logs. [8]
\newline 
\item Access Frequency based Anti-Crawling
\newline 
Access frequency based anti-crawling is a method that determines a client is whether a crawler or a legitimate user by access frequency rate. A web server can set a threshold limit of access count in an unit time. If a client with a specific IP requests exceeds this limit in pre-defined time, the web server determines that this IP as a crowler node. This method has two well known problems. First, it has vulnerability against distributed crawler.[8] If an attackers use distributed crawler service such as Crawlera, access rate for a crawler node IP will be reduced enough to bypass threshold limit. Second, it could raise false positive error if many users share a public IP.
\newline


\end{enumerate}


%
% BLOCKING DISTRIBUTED CRAWLER
% with 2 subsections
%
\section{BLOCKING DISTRIBUTED CRAWLER}
As described in previous section, distributed crawler can bypass every conventional anti-crawling methods. In this section, we propose a new technique to detect and block distributed crawlers that could not be defended by existing anti-crawling techniques.

\subsection{Required Number of Crawler Nodes}
In order for the Distributed crawler to collect the entire data of a website, the following conditions must be met.\\
\begin{displaymath}
Cn \geq Um / (Td * 30) 
\end{displaymath}

The above formula can be described as follows.

\begin{itemize}
\item Um: Number of items u›ated in month
\item Td: Maximum number of request per IP
\item Cn: Number of crawler node(IP)
\end{itemize}

For an example, if there is a web service updates Um(30,000) items in a month and the service has a restriction rule that if an IP requests more then Td(100) times will be banned, than an attacker who tries to collect every items from the service use at least Cn(10) crawler nodes to avoid restriction.\\
Therefore, as Um increases or Td decreases, Cn increases. And Cn numerically indicates the level at which the website is difficult to crawl.

\subsection{Generating Long-tail Td zone}
The number of items that are updated in a month can not be arbitrarily increased. Simple way to prevent distributed crawler is to decrease Td. But this will also increase false positive significantly. In this paper, we solve this problem by reversing the general characteristics of web traffic and the fact that the crawler tries to replicate the entire data.\\
If you sort the items by access rate, you can see the exponentially decreasing form as shown in the following figure. This means most of the web traffic is concentrated on most frequently requested items. And there is a long-tail region that has low access rate. We calculated max request count of this longtail region and set this value as Td3.

\begin{figure}[H]
    \includegraphics[width=0.7\columnwidth]{figs/figure_01.png}
    \caption{Access Frequency per number of connections}
    \label{fig:my_label}
\end{figure}

In information theory, unlikely event is more informative then likely event. And the event that a request to an item in long-tail region is much unlikely then the upper items. This means the web service can find more information from a request to the long-tail region.\\
Hence, when a client keep request the items in long-tail region, the web service increase the count until it reaches Td3, instead of reaching Td mean. This means a web service can set much more sensitive threshold without increasing false positive error rate.\\


\subsection{Node Reducing with Long-tail Region}
In order for an attacker to collect the entire data from the service, he or she must also access the items in the long-tail (Td3) interval. However, the attacker does not know exactly which item the item he is accessing belongs to. Using this information asymmetry, service providers can easily identify IPs that are accessed more frequently than long-tailed segments. These identified crawler IPs will included in block-list and the number of IPs in blcok-list will be called Cm. If we start to increase the Cm value through the long-tail interval, the attacker will crawl with a smaller number of IPs, and Cm will increase in the Td3 interval.

\begin{itemize}
\item Cm: Number of crawler node(IP) blocked by service
\item long\_t: Ratio of items included in long-tail region
\end{itemize}

So attackers must satisfy the following inequality. Cn - Cm is the number of non-blocked crawler nodes and this should be greater then the right term.

  \begin{displaymath}
Cn - Cm \geq (Um * long\_t) / (Td3 * 30)
  \end{displaymath}\newline

On the service provider side, Cm should be greater then right term to block distributed crawlers.

  \begin{displaymath}
Cm > Cn - (Um * long\_t) / (Td3 * 30)
  \end{displaymath}\newline

If a particular IP accesses an item in the long-tail region with more than the Td3 value determined by the above formula, it can be included in the block-list.


\subsection{Dummy Items}
The service provider may include a dummy item to detect the crawler in addition to the actual service target item. Dummy items are inaccessible for the legitmiate user because there are no user interfaces for dummy items or hidden. There are few ways to generate dummy items, it may exists as an HTML tag but it is not displayed on the screen by the attribute setting or it may contains a garbage information that normal users shell not be interested. But a crawler that performs sequential access to the service will may access the dummy items. By this characteristic, dummy items can work as extension of the long-tail region.\\
In this paper, we will not include dummy items in experiments due to fair comparsion with real traffic logs which will never access dummy items.\\



%
% EXPERIMENT
% 1. Web Traffic Data
% 2. Simulation
% 3. Node Reducing Result
%
\section{EXPERIMENT}
Our experiment was designed to verify the classification performance of crawler detection module between crawlers and legitimate  web traffics. We compared LTM(Long-tail Threshold Model) with normal access frequency based anti-crawling on maximum number of crawler nodes and false positive.\\
We used the real web traffic log that NASA released in 1995. And then we performed pre-processing, modeling and simulation.

\begin{figure}[H]
    \centering
    \includegraphics[width=0.85\columnwidth]{figs/figure_design.png}
    \caption{Experiment Design}
    \label{fig:my_label}
\end{figure}


\subsection{Web Traffic Data}
\begin{enumerate}
\item Source
\newline NASA released a total of 1,891,715 access logs for the month of July 1995. We parsed this log into csv format and composed of 4 columns including IP, date, access target and access result.
The total number of connected IPs are 81,978 and the number of items are 21,649.
\newline

\item Data Pre-Processing and Traffic Distribution
\newline In order to prevent duplication of data used for in modeling and experimental data in the time series data, we split the log by time. July 1 to 24 was used for LTM modeling and 25 to 30 was used for testing.
When a user accesses a html file, they also get accesses to gif files that are linked. This can force a multi increase of access count. Hence we removed some gif files from the long-tail. For the last we excluded the request logs that the access results are not success from the experiment. After doing the pre-processing we have got the test set as below table.

\begin{table}[H]
  \caption{Pre-Processed Web Traffic Data}
  \label{tab:freq}
    \begin{tabular}{| p{3.1cm} | p{2cm} |}
    \hline
    Num of Item & 7,649 \\ \hline
    Num of Long-tail & 5,355 \\ \hline
    Mean of Request & 184.76 \\ \hline
    Mean of Long-tail & 1.88 \\ \hline
    \end{tabular}
\end{table}

As a result, we made a pre-processed traffic data set consists of 7,649 items from 21,649 raw data. Which has 5,355 items as a long-tail region. The mean of the total request count was 184.76, and the mean of long-tail region was 1.88. It means simply we can set about 6880\% more sensitive threshold to the crawler detection algorithm.\\ 

\begin{table}[H]
  \caption{Experiment Data}
    \begin{tabular}{| l | l | l | l | l | }
    \hline
    & Ratio & mean & max & count \\ \hline
    Td1 &  > 0.5\% & 21,250 & 76,040 & 38 \\ 
    Td2 & 0.5\% - 30\% & 264 & 7,043 & 2,256 \\
    Td3 & 30\% > & 1.88 & 9 & 5,355 \\ \hline
    \end{tabular}
\end{table}

The key part was the real web traffic will show the power distribution after sorting with access frequency or not. And we could verify the NASA traffic data also has power distribution as shown in Figures 2, 3 and 4.
\end{enumerate}

\begin{figure}[H]
    \centering
    \includegraphics[width=0.85\columnwidth]{figs/figure_02_td1.png}
    \caption{Access Count in Td1}
    \label{fig:my_label}
\end{figure}
Figure 2 shows the group of the most frequently requtestd items. 

\begin{figure}[H]
    \centering
    \includegraphics[width=0.85\columnwidth]{figs/figure_03_td2.png}
    \caption{Access Count in Td2}
    \label{fig:my_label}
\end{figure}

\begin{figure}[H]
    \centering
    \includegraphics[width=0.85\columnwidth]{figs/figure_04_td3.png}
    \caption{Access Count in Long-tail}
    \label{fig:my_label}
\end{figure}

Figure 4 shows the long-tail region of sorted result. We set Td3 threshold value as 20 which is about 2 times larger than the maximum access rate of long-tail region. Setting a Td3 value in LTM has a heuristic part because web services has different purposes and circumstances.

\subsection{Simulation}
In this paper, we implemented simulation for two purpose. One is to check whether it is possible to detect and disable the crawler IP group with LTM, and the other one is to check false positive when input the actual web traffic to LTM.

\begin{figure} [H]
    \includegraphics[width=0.88\columnwidth]{figs/flow_chart_01.png}
    \caption{Crawler Detection Flow}
    \label{fig:my_label}
\end{figure}


\begin{enumerate}
\item Distributed Crawler Detecting Simulation
\newline 
We used python for implementing the LTM simulator. The required parameters are the 1) size of the distributed IP set used by the crawler, 2) the long-tail list, 3) the entire item list, and 4) threshold values used for detection.\\
The LTM simulator watched the Crawler IP Set to access each item by traversing the entire item list, and increased the access count for the IP at each access to the long-tail entry.
When the crawler accesses a long-tail entry, LTM increases the access count of IP. Exceptionally, if a same IP accesses the same item again, LTM does not increase the access count considering that it is not related to the purpose of crawling. When an access count of IP exceeds the threshold value, LTM adds the corresponding IP to the block list. Figure 5 below shows an example of running a crawler using 100 distributed IPs for 7,649 items. The number of long-tail items is 5,355 and the threshold is set to 20. The following figure is a graph of the process of reducing 222 crawler sets on the simulator. \\

\begin{figure}[H]
    \centering
    \includegraphics[width=0.7\columnwidth]{figs/figure_06_nr.png}
    \caption{Number of IPs reduced by detection}
    \label{fig:my_label}
\end{figure}


We can verify that the crawler IP set is gradually reduced until get totally blocked. When the first crawler node IP access count exceeds the Td3 and blocked, node reducing count increases exponentially. This is because when a crawler node blocked, other crawler nodes get more burden and has to access more items.
\newline

\item Node Reducing Result
\newline 
Experiments were performed with threshold set to 20, and the crawler set consisting of 222 nodes was completely detectable and false positives were 1.33 cases per day, which was 0.0312\% of the daily IP number. In below table, we compared the result of LTM with normal FBA(Frequency Based Anti-crawling) method.

\begin{table}[H]
  \caption{Experiment Data}
    \begin{tabular}{| l | l | l | l | l | }
    \hline
    & Threshold & Max Node & False Positive \\ \hline
    LTM & 10 & 420 & 0.13\% \\ 
    LTM & 20 & 222 & 0.03\% \\ 
    FBA & 10 & 561 & 10.68\% \\
    FBA & 20 & 302 & 3.64\% \\ 
    FBA & 80 & 75 & 0.12\% \\ 
    FBA & 110 & 56 & 0.03\% \\ \hline
    \end{tabular}
\end{table}

We can detect more or less number of crawler nodes depending on the number of items the site has or the length of the long tail. Since this simulation is based on old NASA traffic data(1995), total number of items were quite small compared to modern web services. If a service has 10 times more items than our test data and has equal distribution that service could detect more than 2000 crawler nodes.\\
In our experiment, 8 false positives cases occurred in 6 days of data. We removed duplicated IPs from this result and found that 6 IPs were miss recognized as crawler nodes. The number of requests per month from each of these 6 IPs are as follows. 

\begin{figure}[H]
    \centering
    \includegraphics[width=0.85\columnwidth]{figs/figure_fp_compare.png}
    \caption{False Positive}
    \label{fig:my_label}
\end{figure}

\begin{figure}[H]
    \centering
    \includegraphics[width=0.85\columnwidth]{figs/figure_limit_compare.png}
    \caption{Number of Detected Crawlers}
    \label{fig:my_label}
\end{figure}

In figure 8, x-axis is threshold and y-axis is false positive rate(\%). It shows 


\begin{table}[H]
  \caption{IP and domains generated requests}
  \label{tab:freq}
  \begin{tabular}{ccl}
    \toprule
    IP or domain&Number of requests\\
    \midrule
    156.80.168.122 & 117\\
    163.205.180.17 & 564\\
    dwkm206.usa1.com & 167\\
    jalisco.engr.ucdavis.edu & 424\\
    jbiagioni.npt.nuwc.navy.mil & 2124\\
    sputnix.cas.und.nodak.edu & 101\\
  \bottomrule
\end{tabular}
\end{table}


156.80.168.122 and sputnix.cas.und.nodak.edu, which generated relatively few requests, were detected as crawler nodes because the requests of these IPs were concentrated on a certain date, 29.7\% of them were in the long-tail area Of the respondents.

\end{enumerate}



%
% CONCLUSION
%
\section{CONCLUSION}
In this paper, we introduced LTM(Long-tail Threshold Model) a node reducing method that identifies the IP set of distributed crawlers and gradually reduces IP by using the long-tail region.\\
By simulating with the real web traffic data, LTM effectively identified distributed crawler and showed a very low level of false positive error.
Web crawling against the term of web service or impairs quality is a serious security threat. And considering that there are some crawler developer using distributed crawler proxy service for illegal purposes, LTM could improve web service data security.


%
% FUTURE WORKS
%
\section{FUTURE WORKS}
Web traffic generally tends to generate traffic bursts at certain times. [1] Although the experiment of this paper is based on actual traffic log, since the time point of the data used in the experiment is one month, it does not include cases where a new item is added or an issue occurs and a traffic burst occurs.
In order to apply the results of this paper more securely to actual services, it is necessary to study whether the item movement level and threshold value of long-tail area can be maintained based on actual traffic data for traffic burst occurrence cases.




% Bibliography
\nocite{*}
\bibliographystyle{ACM-Reference-Format}
\bibliography{exbib}




%%
% Introduction
%
\section{Introduction}
Web-crawling is used for collecting data in various fields. Some web-crawlers collect data even though the site is preventing crawlers by robot.txt. This can have a serious impact on the availability of the target service. These web-crawlers modify the header value, distribute IP to masquerade as if they are legitimated users to prevent themselves from detection.
It is also prohibited to duplicate an entire dataset, even if the service permits viewing of individual data. However, malicious distributed web-crawlers browse and replicate the entire data of the service.
In this paper, we introduce the anti-crawling methods and its countermeasures, and show that the conventional anti-crawling method cannot defend the distributed crawler. We also introduce a new anti-crawling technique that gradually adds an IP set using a distributed crawler to the black-list.



%
% Back Ground
% with 2 subsections
%
\section{Back Ground}
\begin{enumerate}
\item Http Archive
\newline Http Archive is a file format that stores network logs obtained through a built-in developer tool for web browsers such as Internet Explorer and Chrome. Network logs are stored as a .har extension files with JSON format. The HAR file consists of metadata about the log itself and entries data for request and response contents. Metadata includes information such as browser type, version, and creation time. In the entries section, start time, time required, and request and response data are included for each entry.
The request and response items in the entry contain all the data that can be found through the developer tool. This means that the developer can obtain the same amount of information as the real-time monitoring of the actual site behavior by checking the har logs. Furthermore, by parsing the har logs, the data necessary for crawler development is automatically extracted.
\newline
\item Power Law
\newline 
According to the rule of thumb, when the items are sorted in order of frequency of use, the frequency of use decreases exponentially every time the rank decreases. This is also common for web traffic, and most web traffic is focused on some frequently used items. In this paper, we also introduce a technique for identifying crawler sets using items in the long-tail region, assuming that web traffic follows the power law.
\end{enumerate}



%
% GENERATING CRAWLING LIBRARY:HAR2LIB
% with 2 subsections
%
\section{Generating Crawling Library:HAR2LIB}
In this chapter, we introduce the har2lib package which generates the crawling library code by parsing the har log file as an example of Intelligent Crawler.

\subsection{Parsing HAR Files}
HAR2LIB provides the harlib class. When you create a class object, it loads the har log file and then parses it. Parsing can be divided into exception handling, header analysis, and URL analysis.

\subsection{Generating Crawling Library}
When parsing is complete, HAR2LIB calls the internal \newline\verb|harlib._gen_py()| method to create a python class that contains a crawling method for the site. Next, the request header information is stored in a dict form inside the method so that the HTTP header can be set to be the same as that transmitted from the browser. The generated class will only contain the necessary methods for crawling, and actually implement the business logic that calls it. This is because the crawling data coding part and the crawling scenario implementation part are separate issues.



%
% ANTI-CRAWLING METHODS AND COUNTER SCENARIO
% with 2 subsections
%
\section{ANTI-CRAWLING METHODS AND COUNTER SCENARIO}
Although Harlib does not present the crawling scenario directly, there are features that help create a crawling scenario, such as an automated dealy. In this chapter, we introduce the anti-crawling technique to detect and respond to the crawler and the bypassing technique.

\subsection{Anti-Crawling Methods}
Detecting crawlers that perform excessive crawling on Web services and performing blocking automatically are very important for service management and intellectual property protection. This section introduces the known anti-crawling techniques.



\begin{enumerate}
\item HTTP Header Check
\newline For a typical crawler, do not use the http header used by the browser. The server distinguishes the normal user from the crawler by checking the request header from the client and checking whether the value of User-Agent is normally included.
\newline
\item Access Frequency Recognition
\newline 
An attacker performing an aggressive crawling predefines the core data that the web service wants to collect, and implements a crawler that requests specific data without requesting unnecessary resources. In this case, the web server must process only a large number of consecutive calls to specific resources. For example, suppose you are an attacker attempting to replicate all the data of a Web service that has real estate transfer data for 5.6 million copies in Korea. When the web service retrieves the lot number, it passes the html, js, and css files to the client and dynamically completes the data that fills the html table form with ajax call. The attacker can implement only the ajax call with a script such as python, and obtain the data from the public address list and request data to the server in parallel. If you are a careful attacker, you will not get close to the performance limit of the server, but if not, you will parallelize the collection process to get the data you want as soon as possible.
\newline
\item Access Pattern Recognition
\newline
Access pattern recognition is an anti-crawling method through dynamic analysis in addition to the above access frequency recognition. If a client requests only a specific resource continuously without a call to a resource that should normally be requested, the corresponding IP is blocked. In the case of a web service using advanced approach to access pattern recognition, the service is viewed as a set of consecutive requests from the viewpoint of the user UX, and requests and responses belonging to the same set are chained by including a specific hash value in the cookie. Some requests can not be made separately.
\end{enumerate}


\subsection{Anti-Cralwer Counter Method}
When using the intelligent Crawler library created through har2lib described in the previous chapter 3, the following three anti-crawling methods can be used as follows. The distributed crawler using all of the techniques 1 to 3 can not be defended by the anti-crawling technique described above.



\begin{enumerate}
\item Request Header Replay
\newline All request methods generated by har2lib automatically mount the same http request header as requested by a normal browser. Web services can not distinguish between crawlers and browsers with http header verifications.
\newline
\item Access Frequency Auto Configuration
\newline 
har2lib sets the automated dealy time value for each method using the actual time data for each method in the har parsing process. All methods manipulate the delay, which is actually taken, through the sleep function when the auto dealy option is set to True. After the configuration is applied, it is possible to configure a parallel crawling network consisting of units that do not exceed the access frequency limit per IP when combined with multiple IP proxy servers.
\newline
\item Access Pattern Replay
\newline
Although har2lib does not write the crawling scenario directly, it supports a guide to the access pattern that helps scenarios that only replicate the ajax call clone to bypass the approach pattern. Since the harlog itself records normal access patterns, it is possible to summarize the access patterns in a separate chart.
\end{enumerate}



%
% BLOCKING DISTRIBUTED CRAWLER
% with 2 subsections
%
\section{BLOCKING DISTRIBUTED CRAWLER}
In this section, we propose a new technique to detect and block distributed crawlers that could not be defended by existing anti-crawling techniques.

\subsection{Number of Distributed Nodes}
In order for the Distributed crawler to replicate the historical data of a website, the following conditions must be met. Cn ≥ Um / (Td * 30) for the number of items (Um) updated in the target site on a month, the maximum number of connections (Td) per day restricted by the site, and the number of IPs (Cn) Must be satisfied.
For example, if a service that updates 60,000 data per month limits the maximum number of connections per day to 50, an attacker must perform crawling using at least 40 distributed IPs. Conversely, on the service provider side, the larger the Um, the smaller the Td is, the more advantageous it is. However, Um is difficult to secure arbitrarily, and if Td is reduced, the ratio of false positives to normal users increases.


\subsection{Node Reducing with statistical approach}
Instead of increasing the Um from the service provider side, or reducing the Td is a method of identifying a portion of the crawler to Cn and Cn by reducing the block.
Assuming that the number of IPs used by the attacker is Cm, the attacker must satisfy Cn - Cm ≥ Um / (Td * 30). On the service provider side, Cm satisfying Cm ≥ Cn - Um / (Td * 30) can be obtained.
Service providers can create a block-list without reducing Td in a batch by using statistical techniques. The frequency of access by users is not the same for each item, and when the items with the highest frequency of access are arranged on the left side, they are distributed in a graph form which exponentially decreases according to a power law as follows.

\begin{figure}[H]
    \includegraphics[width=1.0\columnwidth]{figs/figure_01.png}
    \caption{Access Frequency per number of connections}
    \label{fig:my_label}
\end{figure}

In order for an attacker to replicate historical data from the service, he must also access the items in the long-tail (Td3) interval. However, the attacker does not know exactly which item the item he is accessing belongs to. Using this information asymmetry, service providers can easily identify IPs that are accessed more frequently than long-tailed segments. If we start to increase the Cm value through the long-tail interval, the attacker will crawl with a smaller number of IPs, and Cm will increase in the Td2 interval.
The Td value for each interval is calculated by adding the standard deviation (s) of the corresponding interval access frequency to the access frequency value (Amax) of the item having the highest access frequency per IP among the corresponding interval items as follows. 

  \begin{displaymath}
    Td = Amax + s * 2
  \end{displaymath}

If a particular IP accesses an item in the long-tail region with more than the Td value determined by the above formula, it can be included in the block-list.

\subsection{Dummy Items}
The service provider may include a dummy item to detect the crawler in addition to the actual service target item. The item is normally inaccessible to the general user through the UI. For example, it exists as an HTML tag but it is not displayed on the screen due to the attribute setting or the case where the ordinary user is not interested because it exists on the index but is not in the real world.
Such a dummy item may approach a crawler that performs sequential access but it can maintain a relatively low threshold value because the accessibility of the general user is low and it generates a lower interval than the long-tail interval derived from the traffic can do.



%
% EXPERIMENT
%
\section{EXPERIMENT}
In order to verify the above, experiments were performed to classify the crawler IP for the actual web traffic data. It is based on the web traffic of 1 month released by NASA. Details and experimental method of data are as follows.


\subsection{Web Traffic Data}


\begin{enumerate}
\item Source
\newline NASA released a total of 1,891,715 access logs for the month of July 1995. In this paper, the log is parsed into csv format and composed of 4 columns including IP, date, access target and access result.
The total number of connected IPs is 81,978 and the number of items is 21,649. The most accessed items received 111,116 requests as '/images/NASA-logosmall.gif'.
\newline
\item Traffic Distribution
\newline 
The total number of accesses is calculated for each access target, and the sorting is performed in order of the largest number of connections. The results are confirmed to be distributed in a form in which a power law is applied as described in Section 5. 
In addition, as shown in Figures 2, 3 and 4, power distribution is also observed internally in Td1, Td2, and Long-tail sections. Figurue 2 is a graph of connection frequency of 38 items corresponding to the upper 0.5\% of Td1, Figure 3 shows the top 100 ~ 2000 figures corresponding to Td2, Figure 4 shows the frequency of 100 ~ 2000 . In the next section of the simulator, node reduction will be performed using a set of items belonging to the long-tail as described above.
\end{enumerate}

\begin{figure}[H]
    \centering
    \includegraphics[width=0.7\columnwidth]{figs/figure_02_td1.png}
    \caption{Access Count in Td1}
    \label{fig:my_label}
\end{figure}

\begin{figure}[H]
    \centering
    \includegraphics[width=0.7\columnwidth]{figs/figure_03_td2.png}
    \caption{Access Count in Td2}
    \label{fig:my_label}
\end{figure}

\begin{figure}[H]
    \centering
    \includegraphics[width=0.7\columnwidth]{figs/figure_04_td3.png}
    \caption{Access Count in Long-tail}
    \label{fig:my_label}
\end{figure}

\subsection{Simulation}
In this paper, we implemented two kinds of simulation. One is to check whether it is possible to detect and disable the crawler IP group by performing node reduction through items belonging to the long-tail, and the other is to check false positive when the actual traffic is input to the same detection logic .


\begin{enumerate}
\item Data Pre-Processing
\newline In order to prevent duplication of data used in modeling and experimental data in the time series data, Long-tail was constructed by using data from 1 to 24 days in 30 matching data. Traffic verification was performed from the 25th to the last day Data. \& Lt; / RTI \& gt;
When accessing html files, gif extension files are removed from the long-tail group in order to prevent cumulative access values from increasing in duplicate while accessing gif extension files together.
Finally, the request log which was not accessed successfully was excluded from the experiment.
\newline
\item Simulators
\newline 
The simulator is implemented using python. The parameters are the size of the distributed IP set used by the crawler, the long-tail list, the entire item list, and threshold values used for detection.
The implementation method allows the Crawler IP Set to access each item by traversing the entire item list, and accesses the IP in the crawler distributed IP set at each access.
When the crawler accesses a long-tail entry, it adds the IP to the warning dictionary and increments the access count by one. However, if the same IP accesses the same item, it does not increase the access count because it is not related to the purpose of crawling. When the access count exceeds the threshold value, Node Reducing is implemented by adding the corresponding IP to the banned list. Figure 5 below shows an example of running a crawler using 100 distributed IPs for 7,649 items. The number of long-tail items is 5,355 and the threshold is set to 20. 

\begin{figure}[H]
    \includegraphics[width=0.65\columnwidth]{figs/figure_05_td4.png}
    \caption{Number of IPs using crawling}
    \label{fig:my_label}
\end{figure}

The IPs included in the crawler IP set gradually accumulate the access count, and the node reduction starts from the point when the access count of the entire crawler node group increases beyond {number of nodes} * {threshold}.
Another function of the simulator is to input the request request to the crawler simulator based on the actual web traffic log. This is implemented to confirm the case where the simulator judges the actual web traffic as a crawler.
\newline
\item Node Reducing Result
\newline 
Experiments were performed with threshold set to 30, and the crawler set consisting of up to 222 nodes was completely detectable. If the number of nodes exceeded 300, it was not detected at all. False positives were 1.33 cases per day, which was 0.0312\% of the daily average IP number of 3,631. The following figure is a graph of the process of reducing 222 crawler sets on the simulator.

\begin{figure}[H]
    \centering
    \includegraphics[width=0.7\columnwidth]{figs/figure_06_nr.png}
    \caption{Number of IPs reduced by detection}
    \label{fig:my_label}
\end{figure}

This is based on NASA traffic data in 1995, and can be applied to more or less crawler sets depending on the number of items the site has or the length of the long tail.
False positives occurred in 8 out of 6 matching data and 6 IPs were recognized as crawler nodes except duplicate detection. The number of requests generated per month for each IP is as follows.


\begin{table}
  \caption{IP and domains generated requests}
  \label{tab:freq}
  \begin{tabular}{ccl}
    \toprule
    IP or domain&Port\\
    \midrule
    156.80.168.122 & 117\\
    163.205.180.17 & 564\\
    dwkm206.usa1.com & 167\\
    jalisco.engr.ucdavis.edu & 424\\
    jbiagioni.npt.nuwc.navy.mil & 2124\\
    sputnix.cas.und.nodak.edu & 101\\
  \bottomrule
\end{tabular}
\end{table}


156.80.168.122 and sputnix.cas.und.nodak.edu, which generated relatively few requests, were detected as crawler nodes because the requests of these IPs were concentrated on a certain date, 29.7\% of them were in the long-tail area Of the respondents.

\end{enumerate}



%
% CONCLUSION
%
\section{CONCLUSION}
In this paper, we introduce a node reducing method that identifies the IP set of distributed crawlers and gradually reduces IP by using the property that web traffic follows the power law.
The node reducing scheme has shown a very low level of false positives against distributed crawlers using multiple IPs, effectively identifying crawler sets.



%
% FUTURE WORKS
%
\section{FUTURE WORKS}
Web traffic generally tends to generate traffic bursts at certain times. [1] Although the experiment of this paper is based on actual traffic log, since the time point of the data used in the experiment is one month, it does not include cases where a new item is added or an issue occurs and a traffic burst occurs.
In order to apply the results of this paper more securely to actual services, it is necessary to study whether the item movement level and threshold value of long-tail area can be maintained based on actual traffic data for traffic burst occurrence cases.



%
% REFERENCES
%
\section{REFERENCES}
[1]	M.V Simkin and V.P. Roychowdhury, “A theory of web traffic” https://arxiv.org/pdf/0711.1235.pdf
\newline[2] Density Estimation for Statistics and Data Analysis
\newline[3] Explaining World Wide Web Traffic Self-Similarity
\newline[4] Research on Detection Algorithm of WEB Crawler
\newline[5] Design and Implementation of Scalable, Fully Distributed Web Crawler for a Web Search Engine
\newline[6] Design and Implementation of a Distributed Crawler and Filtering Processor
\newline[7] URL Assignment Algorithm of Crawler in Distributed System Based on Hash
\newline[8] Design and Implementation of a High-Performance Distributed Web Crawler
\newline[9] Feature evaluation for web crawler detection with data mining techniques
\newline[10] Crawler Detection: A Bayesian Approach
\newline[11] Real-time Web Crawler Detection
[12] An investigation of web crawler behavior: characterization and metrics


% Bibliography
\bibliographystyle{ACM-Reference-Format}
\bibliography{sambib}




\end{document}
